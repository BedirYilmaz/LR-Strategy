\documentclass{article}
\usepackage[utf8]{inputenc}

\title{LR : Strategy}
\author{3yanlis1bos}
\date{January 2019}

\begin{document}

\maketitle

\section{How to Create an LR document}

This section covers the algorithm for creating atomic LR documents for each paper. We will cover each step of this algorithm in subsections.

\begin{enumerate}
    \item Read
    \item Think
    \item Go away
    \item Return
    \item Check
    \item Cite
\end{enumerate}

\subsection{How to Read Papers Efficiently}

My way of reading papers efficiently is primarily based on "Pete Carr's advices on How to Read Chemistry Papers Efficiently." from YouTube.

First think to know is the order of the sections that we will follow when reading. This is important since we mostly do not possess any information regarding the content of the paper, so our first goal is always to find out whether or not the paper is worth reading.

\begin{enumerate}
    \item Title
    \item Keywords
    \item Skim Through the Paper to see if its bull*** or not
    \item Abstract
    \item Conclusion
    \item Figures and Tables
    \item Methodology
    \item Experimental Results and Discussion
    \item Introduction
\end{enumerate}

We do not take notes on the paper or anywhere else, we do not highlight while reading. It is O.K. to mark however, if there are some words and phrases that you do not understand. That one can actually speed up the process of reading.

The catch of not taking notes on the paper itself is being continuously conscious about the fact that you are going to have to remember what you read. At least until you finish reading the paper. What happens after that will be explained in next subsection.

\subsection{How to Think about a Paper}

You take simple notes. What was the big deal about this paper? What was highlighted? What was hidden so that the reviewers wont be able to understand how bad they screwed up? What did they try to achieve? What was their real major contribution? If they have confessed about the limitations of their approach, what was it? Who did they follow and why? Make sure your notes are valid paraphrases, not quotes.

\subsection{Go Away}

You definitely should go away from your paper and notes for 20 minutes to two hours. This is to leave the influence of the author and your own notes behind and gain a critical perspective about the document. Give your brain a context switch.

\subsection{Return, Check, Cite}

After returning, check your notes and make sure they do not contradict with the truth about the paper. It is also crucial to not to leave anything important behind. Add some more notes if you need.

After that, create a JabRef, Mendeley or EndNote entry as BibTex for the paper. Then apply citations on your notes if needed.

\section{Writing the LR}

In this section, we will discuss the four stages for creating the final LR document. 

\begin{enumerate}
    \item Collect
    \item Compose
    \item Conceptualize
    \item Storify
\end{enumerate}

\subsection{Collect}

At this stage, just collect the paraphrases you have into a single document.

\subsection{Compose}

This stage is about composing themes out of your paraphrases. Mark your paraphrases with theme names and have a print out of the result. At the end, you will have your paraphrases categorized by themes.

\subsection{Conceptualize}

This stage is where you create you Article/Concept Matrix. You will have a number of tables to describe who followed what approach and who addressed which problem.

You also have your Themes in correct order now and paraphrases also ordered under your themes.

\subsection{Storify}

The name speaks for itself. You have to convert the document into a story and do it elegantly. Elegance brings the continuity, flow 

\end{document}
